\documentclass[11pt]{article}
\usepackage{amsmath}
\usepackage{amsfonts}
\usepackage{enumitem}
\usepackage{tikz}
\usepackage{float}
\usepackage[top=1in, left=1in, right=1in, bottom=1in]{geometry}

\setlength{\parindent}{0cm}

\begin{document}

\section{Parameters}

\subsection{Time periods}
Define a set of time periods $\textbf{T} = \{1, 2, \cdots T\}$ and the length of each time period is $\Delta t$ (hour).

\subsection{Battery techno-economic parameters}

Define the following techno-economic parameters of batteries.

\begin{table}[H]
	\centering
	\caption{Techno-economic parameters of batteries.}
	\label{my-label1}
	\begin{tabular}{lll}
		\hline
		Notation & Notes                         & Units \\
		\hline
		$r_C$    & Charge rate                   & kW    \\
		$\delta$ & Healthy depth of discharge    &       \\
		$s_0$    & Battery initial energy level  & kWh   \\
		$\eta$   & Battery round-trip efficiency &       \\
		$\eta_I$ & Inverter efficiency           &       \\
		$C^I$    & Installation cost             & \$    \\
		$C^C$    & Battery capital cost          & \$/kW \\
		$r$      & Discount rate                 &       \\
		$l$      & Battery lifetime              & yrs   \\
		\hline
	\end{tabular}
\end{table}

\subsection{Demand}
Define 15-min load in kW: $d_t,\forall t \in \textbf{T}$.

\subsection{Tariffs}
Assume $n_{DT}$ demand charge rates are included in an energy tariff. Define sets $\textbf{T}_j \subseteq \textbf{T}, \forall j \in \{1,2 \cdots n_{DT} \}$, where each $\textbf{T}_j$ represents one set of time periods associated with the $j$th demand rate.
\begin{enumerate}
	\item Energy charge rate $C^E_t$ (\$/kWh), $\forall t \in \textbf{T}$.
	\item Demand charge rate $C^D_j$ (\$/kW), $\forall j = \{1,2 \cdots n_{DT} \}$.
	\item One-time service charge $C^S$ (\$).
\end{enumerate}

\section{Decision variables}
We define following decision variables in Table~\ref{my-label2}.

\begin{table}[H]
	\centering
	\caption{Decision variables}
	\label{my-label2}
	\begin{tabular}{lll}
		\hline
		Variables                                                    & Units & Notes                                                         \\
		\hline
		$x\in \mathbb{R}_+$                                          & kWh   & Battery capacity                                              \\
		$\overline{d}\in \mathbb{R}_+$                               & kW    & Demand limit                                                  \\
		$y_t\in \mathbb{R}_+, \forall t \in \mathbf{T}$              & kW    & Improved demand                                               \\
		$y^p_j\in \mathbb{R}_+, \forall j \in \{1, \cdots n_{DT} \}$ & kW    & Peak load during period $\mathbf{T}_j$                        \\
		$z^+_t\in \mathbb{R}_+, \forall t \in \mathbf{T}$            & kWh   & Battery charge during period $t$                              \\
		$z^-_t\in \mathbb{R}_+, \forall t \in \mathbf{T}$            & kWh   & Battery discharge during period $t$                           \\
		$z_t\in \mathbb{R}, \forall t \in \mathbf{T}$                & kWh   & Net energy change of battery                                  \\
		$s_t \in \mathbb{R}_+, \forall t \in \mathbf{T}$             & kWh   & Battery energy level at the END of period $t$                 \\
		$b_t \in \{0, 1\}, \forall t \in \mathbf{T}$                 &       & Battery charge/discharge indicator, 0 - charge, 1 - discharge \\
		\hline
	\end{tabular}
\end{table}

\section{Constraints}

\subsection{Load improvement}
\begin{equation}
y_t = d_t + \eta_I \sqrt{\eta} \cdot \frac{z_t}{\Delta t}
\end{equation}

\subsection{Peak load constraint}
\begin{equation}
y^p_j \ge y_t, \forall j=\{1, \cdots n_{DT} \}, \forall t \in \textbf{T}_j
\end{equation}

\subsection{Battery operation constraints}
\begin{equation}
z_t = z^+_t - z^-_t, \forall t \in \textbf{T}
\end{equation}

\subsubsection{Charge/discharge rate}
\begin{eqnarray}
z^+_t \le r_C \Delta t \cdot (1 - b_t), \forall t \in \textbf{T} \\
z^-_t \le r_C \Delta t \cdot b_t, \forall t \in \textbf{T}
\end{eqnarray}

\subsubsection{Healthy depth of discharge}
\begin{equation}
\text{max} \{ 0.05, \frac{1-\delta}{2} \} \cdot x
\le 
s_t 
\le 
\left( 1 -  \frac{1-\delta}{2} \right) \cdot x,
\forall t \in \textbf{T}
\end{equation}

\subsubsection{Energy level continuity}
\begin{equation}
s_t = s_{t-1} + z_t, \forall t \in \textbf{T}
\end{equation}

\subsection{Demand limit rule}
This constraint indicates that the battery can only charge when load is lower than load limit $\overline{d}$, and can only discharge when load is greater than $\overline{d}$.
\begin{eqnarray}
-\inf \cdot b_t
\le
\overline{d} - d_t
\le
\inf \cdot (1-b_t) \\
-\inf \cdot b_t
\le
\overline{d} - y_t
\le
\inf \cdot (1-b_t)
\end{eqnarray}

\section{Objective function}
The objective function $f$ consists of two parts. The first part is monthly cost of battery ($f_1$), while the second part accounts for total tariff charge ($f_2$).

\begin{eqnarray}
\text{min:} & f & = f_1 + f_2 \\
& f_1 & = \frac{C^I + C^C x}{12} \cdot \frac{r (1+r)^l}{(1+r)^l - 1} \\
& f_2 & = C^S + \sum_{t \in \textbf{T}} C^E_t \cdot y_t + \sum_{j=1}^{n_{DT}} C^D_j \cdot y^p_j
\end{eqnarray}

\end{document}